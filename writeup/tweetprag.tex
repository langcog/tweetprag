\documentclass[11pt,letterpaper]{article}
\usepackage{naaclhlt2015}
\usepackage{times}
\usepackage{latexsym}
\setlength\titlebox{6.5cm}    % Expanding the titlebox

% Effects of non-linguistic context
\title{Shared common ground influences information density in microblog texts\Thanks{Thanks to...}}

\author{Gabriel Doyle\\
	    Stanford University\\
	    450 Serra Mall\\
	    Stanford, CA 94305, USA\\
	    {\tt gdoyle@stanford.edu}
	  \And
          Michael C. Frank\\
	    Stanford University\\
	    450 Serra Mall\\
	    Stanford, CA 94305, USA\\
	    {\tt mcfrank@stanford.edu}}

\date{}

\begin{document}
\maketitle
\begin{abstract}


\end{abstract}

\section{Introduction}

Intro to information theoretic views of language \cite{genzel2002}

The role of non-linguistic context. Common ground \cite{clark1996} \cite{brennan1990}

\begin{equation}
H_L+ H_{NL} = C
\end{equation}

Specific research on UID and information rate \cite{qian2012} \cite{levy2007}

Why twitter? Research on twitter

Our contribution here. 



\section{Corpus and Methods}

\subsection{\#worldseries Corpus}

\subsection{Entropy Computation}

\section{Temporal Changes in Information Rate}

Genzel and charniak 2002 reproduction

Transition: but wait, why is this happening?

\section{Non-Linguistic Sources of Information Rate Variance}

Main finding: total entropy related 

Need to discuss length, but focus on entropy?


\section{Control Analyses}

\subsection{Non-Rate Metrics of Context}

\subsection{Speaker Normalization}




\section{Discussion}


\section*{Acknowledgments}

We gratefully acknowledge the support of ONR Grant N00014-13-1-0287.

\bibliographystyle{naaclhlt2015}
\bibliography{tweetprag}

\end{document}
